\documentclass[sigchi-a, authorversion]{acmart}
\usepackage{booktabs} % For formal tables
\usepackage{ccicons}  % For Creative Commons citation icons


\setcopyright{licensedcagov}

\copyrightyear{2019}
\acmYear{2019}
%\setcopyright{acmlicensed}
\acmConference[CHI 2019] {CHI Conference on Human Factors in Computing Systems Proceedings}{May 4--9, 2019}{Glasgow, Scotland UK}
% Note UK should be in caps
\acmBooktitle{CHI Conference on Human Factors in Computing Systems Proceedings (CHI 2019), May 4--9, 2019, Glasgow, Scotland UK}
\acmPrice{15.00}
\acmISBN{978-1-4503-5970-2/19/05}
\acmDOI{10.1145/XXXXXX.XXXXXX}
% Authors, replace the red X's with your assigned DOI string during the rights review eform process.

\newcommand{\ruzica}[1]{{\color{red}\textbf{Ruzica: } #1}}
\newcommand{\markS}[1]{{\color{green}\textbf{Mark: } #1}}
\newcommand{\bill}[1]{{\color{blue}\textbf{Bill: } #1}}

\settopmatter{printacmref=false}

\begin{document}
\title[]{Live Programming By Example} 


%% Author with single affiliation.
\author{Mark Santolucito}
%\authornote{with author1 note}          %% \authornote is optional;
                                        %% can be repeated if necessary
%\orcid{nnnn-nnnn-nnnn-nnnn}             %% \orcid is optional
\affiliation{
  %\position{Position1}
  \department{Computer Science}              %% \department is recommended
  \institution{Yale University}            %% \institution is required
  }
\email{mark.santolucito@yale.edu}          %% \email is recommended

\author{William T. Hallahan}
%\authornote{with author1 note}          %% \authornote is optional;
                                        %% can be repeated if necessary
%\orcid{nnnn-nnnn-nnnn-nnnn}             %% \orcid is optional
\affiliation{
  %\position{Position1}
  \department{Computer Science}              %% \department is recommended
  \institution{Yale University}            %% \institution is required
  }
\email{william.hallahan@yale.edu}          %% \email is recommended

% \author{Jack Huang}
% %\authornote{with author1 note}          %% \authornote is optional;
%                                         %% can be repeated if necessary
% %\orcid{nnnn-nnnn-nnnn-nnnn}             %% \orcid is optional
% \affiliation{
%   %\position{Position1}
%   \department{Computer Science}              %% \department is recommended
%   \institution{Yale University}            %% \institution is required
%   }
% \email{jack.huang@yale.edu}          %% \email is recommended

%% Author with single affiliation.
\author{Ruzica Piskac}
%\authornote{with author1 note}          %% \authornote is optional;
                                        %% can be repeated if necessary
%\orcid{nnnn-nnnn-nnnn-nnnn}             %% \orcid is optional
\affiliation{
  %\position{Position1}
  \department{Computer Science}              %% \department is recommended
  \institution{Yale University}            %% \institution is required
  }
\email{ruzica.piskac@yale.edu}          %% \email is recommended

\begin{abstract}
Live programming is a novel approach for programming practice. Programmers are given real-time feedback when writing code,
traditionally via a graphical user interface. 
Despite live programming's practical values, such as providing an easier overview of code and better understanding of its structure, live programming is still not widely used by everyday programmers.
In this work, we extend the live programming paradigm to general purpose code editors, which allows for
live programming to finally be used by programmers and gives another way of understanding what the given code is doing, and how to easily change it.
To achieve this we extended a fully-featured IDE with the ability to show input/output examples of code execution,
as the programmer is writing code.
Furthermore, we integrate programming by example (PBE) synthesis into our tool by 
allowing the user to change the shown output, and have the code update automatically.
Our goal is to use live programming to give novice programmers a new way to interact and understand programming,
  as well as being a useful development tool for more advanced programmers.
% Programming by example is a powerful approach to program synthesis: the main idea is
% a programmer will provide a few 
% representative examples, indicating her intentions, and then a PBE tool will
% automatically generate code satisfying those examples.
% While the theoretical foundations of the PBE paradigm continue to expand the potential for synthesis, the PBE approach is not widely used by everyday programmers.
% This is in part due to the lack of a native interface to support this new paradigm of programming.
% We propose using live programming to help novice programmers better understand examples as a mode of programming.

\end{abstract}

\maketitle


\section{Introduction}
\label{sec:intro}


Traditionally, writing a program is a relatively static process: a programmer writes some code and, after a successful compilation, can observe and inspect its behavior. If the code does not actually implement the programmer's intentions, they can correct the program and repeat the process.

The live programming paradigm advocates a more dynamic programming cycle that allows the programmer to inspect and understand the code as it is written. As code is written, a user interface gives realtime feedback.  While existing incarnations of live programming environments~\cite{victor2012, chugh2016programmatic, brown2009interacting} mainly focus on programs with graphical or auditory output, we have developed a methodology for general-purpose live programming.
We seek to both give programmers immediate feedback as they write code,
and to allow them to avoid manually writing code altogether.
We accomplish both of these goals through input/output examples.
We introduce \textit{live debugging}, which shows realtime feedback via changing outputs to fixed inputs as a function is modified. We also make use of \textit{programming by example} (PBE), which offers automated repairs based on input/output examples. 
As an implementation, we have developed a Javascript live coding environment as a plugin for the Atom text editor~\cite{Atom}.

% We introduce \textit{live debugging} as a technique to show realtime feedback as programmers write code.
% Programmers can specify an arbitrary number of function inputs,
% which are executed in realtime as they write and modify functions.
% By observing changes in the input-output pairs,
% the user receives immediate feedback about whether the code is correct without actually analyzing it in detail.
% Any example that behaves unexpectedly acts as a real-time indication of an error in the code. 

% Our framework also allows for \textit{programming by example}~\cite{cypher93,lieberman01,synasc12}.
% PBE  is a synthesis technique that automatically generates programs that coincide with given examples. An example is specified as a tuple of input and output values. Given a set $S= \{(i_1, o_1),\ldots, (i_n, o_n)\}$ of input/output examples, the goal is to automatically derive a program $P$ such that for every $j$, $P(i_j) = o_i$. The success and impact of this line of work can be seen from the fact that some of this technology ships as part of the popular Flash Fill feature in Excel 2013~\cite{flashFillPOPL}.

% Live debugging and programming by example naturally complement each other.
% Programming by example makes use of examples as an easily readable and understandable specification. However, even if the synthesized program satisfies all the provided examples, it still might not correspond to the user's intentions. Examples are, by nature, an incomplete specification. However, since live debugging allows a programmer to immediately and continuously see the effects of changes, the user can provide new examples that better illustrate their intentions when synthesis fails. The synthesized program can then be refined with each new example. We call this approach {\emph{cooperative programming}}.

% By combining live debugging and programming by example,
% our methodology offers programmers a useful work environment.
% Live debugging offers rapid feedback as code is written and modified. 
% When the user encounters unexpected output, they have two options.
% The user can go back to the code, detect
% the source of the error, and correct it manually.
% However, they can also adjust that output value directly,
% and rely on programming by example to ensure the program gives the expected output.

% A live programming environment is obtained through a standard
% Haskell REPL (Read Evaluate Print Loop) paradigm. 

% \noindent\fbox{%
%     \parbox{\textwidth}{%
% A small demo of our envisioned approach is available in the following video \\
% $\qquad$\url{https://www.youtube.com/watch?v=w5aI3N4dq2w}.    }%
% }

% \begin{figure}[h!]
% \centering
% \includegraphics[scale=0.5]{tool}
% \caption{A user interface for a live programming environment.}
% \label{fig:tool}
% \end{figure}


% The next question is how to correct the error? The user can go back to the code, detect
% the source of the error and correct it manually. We, additionally, want to offer an 
% automated repair by integrating recent advances in the PBE paradigm into this framework.
% If the user notices that the current program for an input value $i$ returns an incorrect value $o$, then she can adjust that value to specify that her intended program should return $o'$ instead.
% This feedback could allow a tool to automatically synthesize a program which coincides with all given examples (included the modified ones) and which follows the structure of the original code as closely as possible.

\vspace{-8pt}
\section{Live Coding Plugin}
% We have implemented our live programming methodology as a plugin
% for Javascript programming in the Atom text editor~\cite{Atom}.
As shown in Figures~\ref{fig:init} to~\ref{fig:pbe}, our live programming methodology
relies on two panels.
Programmers write code in one panel.
The other panel displays input/output examples,
which are used for both manual debugging,
and programming by example.
% The other panel displays input/output pairs for each function,
% which update in real time.
% By manipulating the displayed output value,
% the programmer can make use of PBE, to automatically update the code.

% \subsection{Live Debugging}
We introduce \textit{live debugging} as a technique to show realtime feedback as programmers write code.
Programmers can specify an arbitrary number of function inputs.
As users write code, we continually run the code on the given inputs. By observing changes in the input-output pairs,
the user receives immediate feedback about whether the code is correct without actually analyzing it in detail.

As Javascript is an interpreted language, running syntactically correct code is fairly straightforward.
Unfortunately, the process of editing code often involves that code
being in a malformed, syntactically incorrect state.
Thus, we only update the displayed output when the code is, in fact,
syntactically valid.
When the user closes the file or editor,
we write the examples to a metadata file.
We reload the examples when the file is opened again.

% \subsection{Programming by Example}
Our framework also allows for \textit{programming by example}~\cite{cypher1991eager,cypher93,lieberman01,synasc12}.
PBE  is a synthesis technique that automatically generates programs that coincide with given input/output examples. An example is specified as a tuple of input and output values. Given a set $S= \{(i_1, o_1),\ldots, (i_n, o_n)\}$ of input/output examples, the goal is to automatically derive a program $P$ such that for every $j$, $P(i_j) = o_i$. The success and impact of this line of work can be seen from the fact that some of this technology ships as part of the popular Flash Fill feature in Excel 2013~\cite{flashFillPOPL}.

When a user modifies an examples output, we update the code to reflect the change.
To synthesize code, we make use of CVC4's Syntax-guided synthesis (or SyGuS) algorithm~\cite{reynolds2017sygus}.
SyGuS is an approach that performs an enumerative search over the space of possible programs,
based on a given grammar.
We draw possible grammatical elements from the existing function implementation and the provided examples.
This both helps ensure that the newly generated code does not stray too far from the programmers original implementation,
and helps constraint the space CVC4 has to search over.
If we fail to find a solution to the SyGus problem, we can iteratively increase the size of the grammar to include elements not present the user-provided code.

\subsection{Implementation}
We have implemented our live programming methodology as a plugin
for Javascript programming in the Atom text editor~\cite{Atom}.
To demonstrate the key ideas, our implementation supports live debugging for programs manipulating strings.

There have been many systems from the program synthesis community that build custom editors for live programming~\cite{Mayer} or support synthesis for domain-specific languages invented by the researchers~\cite{omar2018live}.
A key contribution in our implementation is embedding live programming by example into a language (Javascript) and an editor (Atom) that has a large userbase.
By implementing our tool in this way, we hope to learn how users interact with live programming by example in the wild.
We can collect logs of synthesis tasks requested by users of the tool to contribute new synthesis benchmarks (for example to the SyGuS competition set~\cite{alur2017sygus}) that more accurately reflect the synthesis tasks that users need.


% \section{Implementation}
We have implemented our live programming methodology as a plugin for the Atom text editor~\cite{Atom}.

\subsection{Synthesis}
When a user modifies an examples output, we update the code to reflect the change.
To synthesize code, we make use of CVC4's Syntax-guided synthesis (or SyGuS) algorithm~\cite{reynolds2017sygus}.
SyGuS is an approach that performs an enumerative search over the space of possible programs,
based on a given grammar.
We draw possible grammatical elements from the existing function implementation.
This both helps ensure that the newly generated code does not stray too far from the programmers original implementation,
and helps constraint the space CVC4 has to search over.
If we fail to find a solution to the SyGus problem, we iteratively increase the size of the grammar~\bill{Mark, more about how this is done?}.

% \section{Related Work}
\label{sec:live}

In 2012 Bret Victor gave a presentation \cite{bretVictorVideo} at the Canadian University Software Engineering Conference (CUSEC 2012) that immediately went viral. The main message of his talk was that ``creators need an immediate connection to what they create''.
%Today's programming practice is far from that ideal: we first write code, then we compile it, and only then can we test it. If we are not happy with our program's output, we debug, modify the code and repeat the process.
Victor proposed a new approach to programming in his talk. Dubbed  {\emph {live programming}}, the approach would help programmers better understand the code that they are writing. Similar to WSIWYG text editors, their changes to their code would have immediately observable effects on its output. His influential talk sparked interest for live programming and led to a number of nicely annotated editors for live coding. However, in a follow-up paper \cite{victor2012} Victor stresses that the goal of live programming is not merely to have a fancy editor, but to help the developer better understand what the code is doing.

Live programming is not a new term: as a concept it was mainly used, until recently, in processing images and sound (under the name ``live coding''). However, the ideas of live programming have been present for a long time in the programming language literature under various different names. Hancock studied ``real-time programming'' \cite{HancockPhDThesis}, a live programming environment for helping children to learn how to program. His research was focused on developing a Logo-like programming environment for robotics. Hancock argues that ``the coordination of discrete and continuous process should be considered a central Big Idea in programming and beyond'' \cite{HancockPhDThesis}.

McDirmid, et. al. \cite{McDirmid13oopsla, BurckhardtFHMMTK13_PLDI, McDirmidE14} sparked renewed academic interest in live programming. Their approach combines editing and debugging to obtain the required level of interactivity and  feedback. However, the applications they consider are restricted to graphical user interfaces. They, too, claim that live programming is the direction that  future programming environments should follow: ``Live programming is emerging as the next big step in programming environments that will finally allow us to move beyond our Smalltalk-era IDEs into a more programmer-friendly future. However, existing live programming experiences are still not very useful - they dazzle us with live feedback but that feedback does not really help us write code!'' \cite{McDirmid13oopsla}.

Some forms of live programming are present in various programming environments under different names. These include interactive programming, just in time programming, conversational programming, and on-the-fly-programming. Perera \cite{Perera08Journal} introduces a programming model called ``declarative interaction'' where he develops  a philosophical model that should unify all the live programming concepts. A list of requirements that a system built on this model should support includes: (1) an interaction between code and data; (2) robustness: a program should be able to repair and reconfigure without restarting, (3) transparency: changes and updates in code behavior should be consistent with program semantics, and the system should be able to generate and provide provenance data explaining the changes; (4) modular: in order to ensure scalability of the approach, system interactions should be restricted within a given scope. However, this scope should be able to dynamically change. Despite his description of such a model, Perera concludes that we still ``lack a simple and coherent paradigm for building robust interactive systems''.

Such a system can help the user to write better code, and through continuous interaction the user will gain a more comprehensive understanding of their program. We believe that enriching a live programming paradigm with program synthesis concepts, such as programming by example, will accomplish all of the goals set by Perera's project.


\section{Conclusion}
By combining live debugging and programming by example,
our methodology offers programmers a useful work environment.
Live debugging offers rapid feedback as code is written and modified. 
When the user encounters unexpected output, they have two options.
The user can go back to the code, detect
the source of the error, and correct it manually.
However, they can also adjust that output value directly,
and rely on programming by example to ensure the program gives the expected output.


\bibliographystyle{ACM-Reference-Format}
\bibliography{sources}


\end{document}
