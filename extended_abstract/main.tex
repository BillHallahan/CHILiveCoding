\documentclass[sigchi-a, authorversion]{acmart}
\usepackage{booktabs} % For formal tables
\usepackage{ccicons}  % For Creative Commons citation icons


\setcopyright{licensedcagov}

\copyrightyear{2019}
\acmYear{2019}
%\setcopyright{acmlicensed}
\acmConference[CHI 2019] {CHI Conference on Human Factors in Computing Systems Proceedings}{May 4--9, 2019}{Glasgow, Scotland UK}
% Note UK should be in caps
\acmBooktitle{CHI Conference on Human Factors in Computing Systems Proceedings (CHI 2019), May 4--9, 2019, Glasgow, Scotland UK}
\acmPrice{15.00}
\acmISBN{978-1-4503-5970-2/19/05}
\acmDOI{10.1145/XXXXXX.XXXXXX}
% Authors, replace the red X's with your assigned DOI string during the rights review eform process.

\newcommand{\ruzica}[1]{{\color{red}\textbf{Ruzica: } #1}}
\newcommand{\markS}[1]{{\color{green}\textbf{Mark: } #1}}
\newcommand{\bill}[1]{{\color{blue}\textbf{Bill: } #1}}

\settopmatter{printacmref=false}

\begin{document}
\title[]{Live Programming By Example} 


%% Author with single affiliation.
\author{Mark Santolucito}
%\authornote{with author1 note}          %% \authornote is optional;
                                        %% can be repeated if necessary
%\orcid{nnnn-nnnn-nnnn-nnnn}             %% \orcid is optional
\affiliation{
  %\position{Position1}
  \department{Computer Science}              %% \department is recommended
  \institution{Yale University}            %% \institution is required
  }
\email{mark.santolucito@yale.edu}          %% \email is recommended

\author{William T. Hallahan}
%\authornote{with author1 note}          %% \authornote is optional;
                                        %% can be repeated if necessary
%\orcid{nnnn-nnnn-nnnn-nnnn}             %% \orcid is optional
\affiliation{
  %\position{Position1}
  \department{Computer Science}              %% \department is recommended
  \institution{Yale University}            %% \institution is required
  }
\email{william.hallahan@yale.edu}          %% \email is recommended

% \author{Jack Huang}
% %\authornote{with author1 note}          %% \authornote is optional;
%                                         %% can be repeated if necessary
% %\orcid{nnnn-nnnn-nnnn-nnnn}             %% \orcid is optional
% \affiliation{
%   %\position{Position1}
%   \department{Computer Science}              %% \department is recommended
%   \institution{Yale University}            %% \institution is required
%   }
% \email{jack.huang@yale.edu}          %% \email is recommended

%% Author with single affiliation.
\author{Ruzica Piskac}
%\authornote{with author1 note}          %% \authornote is optional;
                                        %% can be repeated if necessary
%\orcid{nnnn-nnnn-nnnn-nnnn}             %% \orcid is optional
\affiliation{
  %\position{Position1}
  \department{Computer Science}              %% \department is recommended
  \institution{Yale University}            %% \institution is required
  }
\email{ruzica.piskac@yale.edu}          %% \email is recommended

\begin{abstract}
Live programming is a novel approach for programming practice. Programmers are given real-time feedback when writing code, traditionally via a graphical user interface. Despite live programming's practical values, such as providing an easier overview of code and better understanding of its structure, it is not yet widely used. In this work, we extend live programming to general purpose code editors, which allows for live programming to be used by programmers, and provides new interfaces for understanding and changing the functionality of code. To achieve this we extended a fully-featured IDE with the ability to show input/output examples of code execution, as the programmer is writing code. Furthermore, we integrate programming by example (PBE) synthesis into our tool by allowing the user to change the shown output, and have the code update automatically. Our goal is to use live programming to give novice programmers a new way to interact and understand programming, as well as being a useful development tool for more advanced programmers.
% Programming by example is a powerful approach to program synthesis: the main idea is
% a programmer will provide a few 
% representative examples, indicating her intentions, and then a PBE tool will
% automatically generate code satisfying those examples.
% While the theoretical foundations of the PBE paradigm continue to expand the potential for synthesis, the PBE approach is not widely used by everyday programmers.
% This is in part due to the lack of a native interface to support this new paradigm of programming.
% We propose using live programming to help novice programmers better understand examples as a mode of programming.

\end{abstract}

\maketitle


\section{Introduction}
\label{sec:intro}


Traditionally, writing a program is a relatively static process: a programmer writes some code and, after a successful compilation, can observe and inspect its behavior. If the code does not actually implement the programmer's intentions, they can correct the program and repeat the process.

\begin{marginfigure}
	\setlength{\abovecaptionskip}{0.1pt plus 0.1pt minus 0.1pt}
	\includegraphics[width=0.55\textwidth]{figures/initial_greet}
	\caption{Code is written in the left hand panel,
	while examples are shown in the right hand panel.}
	\label{fig:init}
\end{marginfigure}
\begin{marginfigure}
	\setlength{\abovecaptionskip}{0.1pt plus 0.1pt minus 0.1pt}
	\includegraphics[width=0.55\textwidth]{figures/manual_change}
	\caption{When the code is modified, the examples update in real time.
	Here, the user has added a space to the output, by editing the code.}
	\label{fig:man_change}
\end{marginfigure}
\begin{marginfigure}
	\setlength{\abovecaptionskip}{0.1pt plus 0.1pt minus 0.1pt}
	\includegraphics[width=0.55\textwidth]{figures/pbe_before}
	\includegraphics[width=0.55\textwidth]{figures/pbe_after}
	\caption{The user can also modify the output examples, to repair the code.  Here, the
	user has added a exclamation point to the end of the example's output, resulting in new code that appends an exclamation point.  The old code is preserved in a comment.}
	\label{fig:pbe}
\end{marginfigure}

Of course, this cycle often extends over a long period of time.
Not all bugs are discovered immediately,
and old code often has to be updated to suit new purposes.
Unfortunately, documentation is often incorrect or out of date,
leaving the best resource for developers as the code itself~\cite{latoza2006maintaining}.
In fact, it is estimated that half of a programmers time
is spent just \textit{comprehending} previously written code~\cite{corbi1989program}.

The live programming paradigm advocates a more dynamic programming cycle that allows the programmer to inspect and understand the code as it is written. As code is written, the user interface gives real-time feedback.  While existing live programming environments~\cite{victor2012, chugh2016programmatic, brown2009interacting} focus on programs with graphical or auditory output, we focus on general-purpose live programming.

We seek to give programmers immediate feedback as they write code,
an understanding of code that was written in the past, 
and a way to avoid manually writing code altogether.
We accomplish these goals through input/output examples.
We use \textit{live debugging} to show realtime feedback via changing outputs to fixed inputs as a function is modified.
We also leverage recent advances in \textit{programming by example} (PBE) to offer automated repairs based on input/output examples.
Programming by example can be a useful technique for novice programmers, for example in a classroom setting~\cite{Suzuki2017}.

Both live debugging and PBE are also beneficial when \textit{later} modifying the code.
As opposed to traditional documentation, the live debugging examples update automatically, and thus will never fall out of date.
PBE simplifies modifying legacy code, by allowing programmers to simply demonstrate new behavior.
Of course, the programmer must be careful that desired existing behavior is preserved,
but this can be done by comparing the old and new code,
rather than having to write new code manually.

% Bill: I'm cutting this because, on rereading, I don't think it actually makes sense.  The examples for live debugging are exactly what is used in synthesis, so you're not going to discover an error by looking at them.  The synthesized program will never change them.
% PBE makes use of examples as an easily readable and understandable specification. However, even if the synthesized program satisfies all the provided examples, it still might not correspond to the user's intentions. Examples are, by nature, an incomplete specification. However, since live debugging allows a programmer to immediately and continuously see the effects of changes, the user can provide new examples that better illustrate their intentions when synthesis fails. The synthesized program is then refined with each new example. We call this approach {\emph{cooperative programming}}.

As an implementation, we developed a Javascript live coding plugin for the Atom text editor~\cite{Atom}.

% We introduce \textit{live debugging} as a technique to show realtime feedback as programmers write code.
% Programmers can specify an arbitrary number of function inputs,
% which are executed in realtime as they write and modify functions.
% By observing changes in the input-output pairs,
% the user receives immediate feedback about whether the code is correct without actually analyzing it in detail.
% Any example that behaves unexpectedly acts as a real-time indication of an error in the code. 

% Our framework also allows for \textit{programming by example}~\cite{cypher93,lieberman01,synasc12}.
% PBE  is a synthesis technique that automatically generates programs that coincide with given examples. An example is specified as a tuple of input and output values. Given a set $S= \{(i_1, o_1),\ldots, (i_n, o_n)\}$ of input/output examples, the goal is to automatically derive a program $P$ such that for every $j$, $P(i_j) = o_i$. The success and impact of this line of work can be seen from the fact that some of this technology ships as part of the popular Flash Fill feature in Excel 2013~\cite{flashFillPOPL}.

% Live debugging and programming by example naturally complement each other.
% Programming by example makes use of examples as an easily readable and understandable specification. However, even if the synthesized program satisfies all the provided examples, it still might not correspond to the user's intentions. Examples are, by nature, an incomplete specification. However, since live debugging allows a programmer to immediately and continuously see the effects of changes, the user can provide new examples that better illustrate their intentions when synthesis fails. The synthesized program can then be refined with each new example. We call this approach {\emph{cooperative programming}}.

% By combining live debugging and programming by example,
% our methodology offers programmers a useful work environment.
% Live debugging offers rapid feedback as code is written and modified. 
% When the user encounters unexpected output, they have two options.
% The user can go back to the code, detect
% the source of the error, and correct it manually.
% However, they can also adjust that output value directly,
% and rely on programming by example to ensure the program gives the expected output.

% A live programming environment is obtained through a standard
% Haskell REPL (Read Evaluate Print Loop) paradigm. 

% \noindent\fbox{%
%     \parbox{\textwidth}{%
% A small demo of our envisioned approach is available in the following video \\
% $\qquad$\url{https://www.youtube.com/watch?v=w5aI3N4dq2w}.    }%
% }

% \begin{figure}[h!]
% \centering
% \includegraphics[scale=0.5]{tool}
% \caption{A user interface for a live programming environment.}
% \label{fig:tool}
% \end{figure}


% The next question is how to correct the error? The user can go back to the code, detect
% the source of the error and correct it manually. We, additionally, want to offer an 
% automated repair by integrating recent advances in the PBE paradigm into this framework.
% If the user notices that the current program for an input value $i$ returns an incorrect value $o$, then she can adjust that value to specify that her intended program should return $o'$ instead.
% This feedback could allow a tool to automatically synthesize a program which coincides with all given examples (included the modified ones) and which follows the structure of the original code as closely as possible.

\vspace{-8pt}
\section{Live Coding Plugin}
% We have implemented our live programming methodology as a plugin
% for Javascript programming in the Atom text editor~\cite{Atom}.
As shown in Figures~\ref{fig:init} to~\ref{fig:pbe}, our live programming methodology
relies on two panels.
Programmers write code in one panel.
The other panel displays input/output examples,
which are used for both manual debugging,
and programming by example.
% The other panel displays input/output pairs for each function,
% which update in real time.
% By manipulating the displayed output value,
% the programmer can make use of PBE, to automatically update the code.

% \subsection{Live Debugging}
We introduce \textit{live debugging} as a technique to show realtime feedback as programmers write code.
Programmers can specify an arbitrary number of function inputs.
As users write code, we continually run the code on the given inputs. By observing changes in the input-output pairs,
the user receives immediate feedback about whether the code is correct without actually analyzing it in detail.

As Javascript is an interpreted language, running syntactically correct code is fairly straightforward.
Unfortunately, the process of editing code often involves that code
being in a malformed, syntactically incorrect state.
Thus, we only update the displayed output when the code is, in fact,
syntactically valid.
When the user closes the file or editor,
we write the examples to a metadata file.
We reload the examples when the file is opened again.

% \subsection{Programming by Example}
Our framework also allows for \textit{programming by example}~\cite{cypher1991eager,cypher93,lieberman01,synasc12}.
PBE  is a synthesis technique that automatically generates programs that coincide with given input/output examples. An example is specified as a tuple of input and output values. Given a set $S= \{(i_1, o_1),\ldots, (i_n, o_n)\}$ of input/output examples, the goal is to automatically derive a program $P$ such that for every $j$, $P(i_j) = o_i$. The success and impact of this line of work can be seen from the fact that some of this technology ships as part of the popular Flash Fill feature in Excel 2013~\cite{flashFillPOPL}.

When a user modifies an examples output, we update the code to reflect the change.
To synthesize code, we make use of CVC4's Syntax-guided synthesis (or SyGuS) algorithm~\cite{reynolds2017sygus}.
SyGuS is an approach that performs an enumerative search over the space of possible programs,
based on a given grammar.
We draw possible grammatical elements from the existing function implementation and the provided examples.
This both helps ensure that the newly generated code does not stray too far from the programmers original implementation,
and helps constraint the space CVC4 has to search over.
If we fail to find a solution to the SyGus problem, we can iteratively increase the size of the grammar to include elements not present the user-provided code.

\subsection{Implementation}
We have implemented our live programming methodology as a plugin
for Javascript programming in the Atom text editor~\cite{Atom}.
To demonstrate the key ideas, our implementation supports live debugging for programs manipulating strings.

There have been many systems from the program synthesis community that build custom editors for live programming~\cite{Mayer} or support synthesis for domain-specific languages invented by the researchers~\cite{omar2018live}.
A key contribution in our implementation is embedding live programming by example into a language (Javascript) and an editor (Atom) that has a large userbase.
By implementing our tool in this way, we hope to learn how users interact with live programming by example in the wild.
We can collect logs of synthesis tasks requested by users of the tool to contribute new synthesis benchmarks (for example to the SyGuS competition set~\cite{alur2017sygus}) that more accurately reflect the synthesis tasks that users need.


% \section{Implementation}
We have implemented our live programming methodology as a plugin for the Atom text editor~\cite{Atom}.

\subsection{Synthesis}
When a user modifies an examples output, we update the code to reflect the change.
To synthesize code, we make use of CVC4's Syntax-guided synthesis (or SyGuS) algorithm~\cite{reynolds2017sygus}.
SyGuS is an approach that performs an enumerative search over the space of possible programs,
based on a given grammar.
We draw possible grammatical elements from the existing function implementation.
This both helps ensure that the newly generated code does not stray too far from the programmers original implementation,
and helps constraint the space CVC4 has to search over.
If we fail to find a solution to the SyGus problem, we iteratively increase the size of the grammar~\bill{Mark, more about how this is done?}.

% \input{related}
\section{Conclusion}
By combining live debugging and programming by example,
our methodology offers programmers a useful work environment.
Live debugging offers rapid feedback as code is written and modified. 
When the user encounters unexpected output, they have two options.
The user can go back to the code, detect
the source of the error, and correct it manually.
However, they can also adjust that output value directly,
and rely on programming by example to ensure the program gives the expected output.


\bibliographystyle{ACM-Reference-Format}
\bibliography{sources}


\end{document}
